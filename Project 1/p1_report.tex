\documentclass[reprint,english,notitlepage]{revtex4-1}


\usepackage[utf8]{inputenc}
\usepackage[english]{babel}

\usepackage{physics,amssymb}  % mathematical symbols (physics imports amsmath)
\usepackage{graphicx}         % include graphics such as plots
\usepackage{xcolor}           % set colors
\usepackage{hyperref}         % automagic cross-referencing (this is GODLIKE)
\usepackage{tikz}             % draw figures manually
\usepackage{listings}         % display code
\usepackage{subfigure}        % imports a lot of cool and useful figure commands

\hypersetup{
    colorlinks,
    linkcolor={red!50!black},
    citecolor={blue!50!black},
    urlcolor={blue!80!black}}

\lstset{
	inputpath=,
	backgroundcolor=\color{white!88!black},
	basicstyle={\ttfamily\scriptsize},
	commentstyle=\color{magenta},
	language=Python,
	morekeywords={True,False},
	tabsize=4,
	stringstyle=\color{green!55!black},
	frame=single,
	keywordstyle=\color{blue},
	showstringspaces=false,
	columns=fullflexible,
	keepspaces=true}


%% USEFUL LINKS:
%%
%%   UiO LaTeX guides:        https://www.mn.uio.no/ifi/tjenester/it/hjelp/latex/ 
%%   mathematics:             https://en.wikibooks.org/wiki/LaTeX/Mathematics

%%   PHYSICS !                https://mirror.hmc.edu/ctan/macros/latex/contrib/physics/physics.pdf

%%   the basics of Tikz:       https://en.wikibooks.org/wiki/LaTeX/PGF/TikZ
%%   all the colors!:          https://en.wikibooks.org/wiki/LaTeX/Colors
%%   how to draw tables:       https://en.wikibooks.org/wiki/LaTeX/Tables
%%   code listing styles:      https://en.wikibooks.org/wiki/LaTeX/Source_Code_Listings
%%   \includegraphics          https://en.wikibooks.org/wiki/LaTeX/Importing_Graphics
%%   learn more about figures  https://en.wikibooks.org/wiki/LaTeX/Floats,_Figures_and_Captions
%%   automagic bibliography:   https://en.wikibooks.org/wiki/LaTeX/Bibliography_Management  (this one is kinda difficult the first time)
%%   REVTeX Guide:             http://www.physics.csbsju.edu/370/papers/Journal_Style_Manuals/auguide4-1.pdf
%%
%%   (this document is of class "revtex4-1", the REVTeX Guide explains how the class works)

\begin{document}
\title{Title}   
\author{Candidate: }               
\date{\today}                             
\noaffiliation                            
\begin{abstract}                          
%In the abstract, we want you to briefly explain, in your words, the objective of the project, how you performed it and the most important results.
%Summary of the report. State the major points and conclusions.



\end{abstract}
\maketitle
\tableofcontents
\section{Introduction}
%Explain in your words the background for the project and how this relates to the objective. Try to relate the project to other phenomena that are common knowledge or that you have obtained during your studies. The introduction is a good place to present equations that are central to the topic. All equations should be pre-sented in the text and all symbols explained.
%Present some background on the topic and explain the problem and why it is relevant. State the outline of the report and what the reader is expected to find in the different sections.


\section{Theory}
%This is where you fill in all the necessary analytical and numerical theory you used. Do not include long derivations, as they will most likely fit best in an appendix. You may however include the result of a derivation, if it is needed. Describe the theory with your own words, to show that you have understood it.
\subsection{Equations and constants}


\subsection{Probability and statistics}


\subsection{(Different models used)}


\subsection{(Comments on appendix material)}


\section{Method (data)}
%Describe all experimental equipment and software that was used to produce the data
%and graphs in the report.
%Describe the observations and how they were taken. State the process you followed for the fitting and for estimating the quantities of interest. Note that you do not need to upload nor describe in detail the programming code you used. But, mention the packages, extra files etc. used. Include also an overview of the algorithm you use in your code.



\section{Results}
%This is where you present tasks on model development and analytical and numerical calculations. Results from simulation and experiments are presented.
%Show the data, the analysis and the results of the analysis. Describe and reference in the text each plot and table that you show.
\subsection{Results from condition 1}


\subsection{Results from condition 2}


\subsection{Results with slight change in condition 1}



\section{Discussion}
%Analysis of experimental or simulation data (your own or given in the assignment) should be discussed in the context of the introductory text and theory.
%Interpretation of the results. Analyze in depth the results of and whether your findings are supported by literature in the context of astrophysical background, and what implications these may have. Identify sources of error and possible improvements.
\subsection{Critical questions}


\subsection{Changing parameters to see what happens}


\subsection{Describing an optional case}


\section{Conclusion}
%A short conclusion where you summarize your results.
%Brief summary of what was covered within the report, and state whether the aim was achieved. Conclude.


\section*{Appendix A: Important calculations}
\subsection*{1. Calculation 1}


\subsection*{2. Calculation 2}


\subsection*{3. Calculation 3}




\onecolumngrid
\vspace{1cm}
\begin{thebibliography}{9}
%All relevant material and literature that you used throughout the process.
\bibitem{xxxx}

\end{thebibliography}



\end{document}